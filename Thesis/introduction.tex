\chapter{Introduction}

\section{Background}

Social enterprises are rising in prevalence in developing countries to help address issues related to social, economic, environmental, and other issues that these countries face. We worked with two different social enterprises in Nepal and India, respectively, and saw the life-changing work they do for extremely disadvantaged people. However, we also encountered some of the problems that these types of organizations face, especially concerning training of employees. Many factors contribute to the difficulty social enterprises have in providing proper and effective training for their employees, including fewer resources relative to other types of organizations and logistical barriers inherent to working in developing countries. Employee training is vital for social enterprises to succeed in both their social missions and business goals, and cannot be taken for granted in the context of the developing world.

Social enterprises take on various approaches to training their employees, but most of these approaches do not overcome the barriers described above. Anudip, the organization Melissa worked with in India, holds annual Training of Trainers events, where hundreds of trainers gather and receive formal lessons on how to teach new employees either English/workplace readiness or information technology (IT) skills. While this event is great for uniting all of the Anudip trainers and sharing ideas, it is not necessarily the most effective way to train employees. The training was impersonal due to such a large number of people, and many of the employees seemed unfamiliar with the material that they were supposed to teach to others. 

Anudip also has informal training in each of its centers, which typically is conducted by having the trainer working at a computer and explaining the material to a group of employees crowded around her. Anudip, like many social enterprises, does not have the resources for designated classrooms, so training happens in the main workspace alongside other working employees. Additionally, this informal training, while very personal, can be fast-paced and overcrowded, making it hard for employees to learn the material effectively. Our senior design project addresses these difficulties social enterprises face in trying to train employees.

Our solution to this issue of training in social enterprises is to use mobile phones, which are becoming increasingly popular in developing countries. We have developed Text to Learn as a system for trainers within social enterprises to upload digital training materials to a common repository. This information can then be sent as SMS text messages to the mobile phones of employees who are registered by the trainers. Text to Learn is interactive with the addition of quizzes created by the trainers. Registered employees can respond to these quizzes via SMS. 

This product addresses the issue of lack of resources by utilizing a resource that nearly all people, even in developing countries, already have--the mobile phone. More specifically, we have developed the system to be used on feature phones, which are less advanced than smartphones and much more common in the developing world. Furthermore, this product overcomes the logistical barriers many social enterprises face by allowing training and learning of materials to be done on employees’ own time, at their own pace, and in their own space. Employees will have full access to training materials as they will be stored on their personal feature phones. They will also not be limited to training only at work where computers are available, but can learn from the materials on their phones at home as well. 

Ultimately, Text to Learn is an affordable, adaptable, accessible, and appropriate solution for social enterprises to provide vital training to their customers and employees and continue producing social benefit for the developing world.


\section{Project Overview}
Our system is a website interface for social enterprises to upload and send training materials and quizzes, which are sent as SMS messages to trainees’ mobile phones. Through the website, trainers are also able to manage users, keep track of all messages sent and received, and monitor users' progress on training and quizzes. Text to Learn could be used in addition to or in place of other traditional training methods.
