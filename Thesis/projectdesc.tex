\chapter{Design and Implementation}

\section{Project Overview}
Our system is a website built using RapidSMS, a web framework, that allows social enterprises to send out training materials to registered users via SMS and monitor users’ progress through SMS-based quizzes.

\section{Requirements}
From discussions with a social entrepreneur, CEO of Anudip Dr. Radha Basu, and from our own experiences working with social enterprises, we have compiled the following list of requirements for this project: \\[-0.7\baselineskip]

\underline{Functional}\\
\textit{Critical}
\begin{itemize}
\item Distributors are able to make training materials available to trainees digitally.
\item Distributors are able to test trainees’ knowledge of training materials.
\item Distributors are able to view results of quizzes.
\item Distributors are able to enable/disable quizzes.
\item Distributors are able to register/unregister trainees for use of system.
\item Trainees are able to access training materials.
\item Trainees are able to receive training materials in small pieces.
\end{itemize}

\textit{Recommended}
\begin{itemize}
\item Distributors are able to send notifications when new training materials are added.
\item Trainees are able to automatically receive results from quiz.
\item Trainees are able to view their progress.
\item Trainees are able to access notifications/training materials multiple times.
\end{itemize}

\textit{Suggested}
\begin{itemize}
\item Distributors are able to set a time limit on the completion of quizzes.\\[-0.5\baselineskip]
\end{itemize}

\underline{Non-Functional}\\*
\textit{Critical}
\begin{itemize}
\item Compatible with basic/feature phones.
\item Easy to read.
\item Low distribution costs for social enterprises.
\item Portable to different web and mobile platforms.
\item Maintainable.
\end{itemize}

\textit{Recommended}
\begin{itemize}
\item Scalable for more users.
\end{itemize}

\textit{Suggested}
\begin{itemize}
\item Open source, extensible.
\item Support for multiple languages.
\end{itemize}

\section{Use Cases}
For our implementation, we have come up with several key use cases for the two types of users of our system: distributors (or trainers) and trainees. The following figures visually describe the main functions for each type of user. The primary use cases are also described in greater detail below.

\underline{Case 1 - Upload Materials}\\*
\textit{Actor:} Distributor\\*
\textit{Goal:} Materials with related quiz are ready to distribute.\\*
\textit{Preconditions:} Have materials in a plaintext format, be a registered user, and internet connection.\\*
\textit{Postconditions:} Materials are uploaded and a quiz is created.\\*
\textit{Scenario:}
\begin{enumerate}
	\item{Navigate to the “Add New Materials” page.}
	\item{Enter text of training materials into the box.}
	\item{Write in the questions answers in the appropriate boxes.}
	\item{Hit Submit.}
\end{enumerate}
\textit{Exceptions:}
\begin{enumerate}
	\item{Text is too long to process.}
	\item{Something on the site is broken.}\\
\end{enumerate}

\underline{Case 2 - Add Users}\\*
\textit{Actor:} Distributor\\*
\textit{Goal:} Grant access to training materials to their employees.\\*
\textit{Preconditions:} Be a registered user, know the phone-number the employee will be using.\\*
\textit{Postconditions:} A registered user can download training materials.\\*
\textit{Scenario:}
\begin{enumerate}
	\item{Navigate to the “Manage Users” page.}
	\item{Select “Add New” by the list of users.}
	\item{Enter the new User’s mobile number and additional information such as name and trainee group.}
	\item{Hit submit to store the number and send a verification text to the trainee.}
\end{enumerate}
\textit{Exceptions:}
\begin{enumerate}
	\item{Incorrect mobile number.}\\
\end{enumerate}

\underline{Case 3 - Manage/Edit Training Materials}\\*
\textit{Actor:} Distributor\\*
\textit{Goal:} Ability to view and edit all uploaded training materials, quizzes, and related SMS messages.\\*
\textit{Preconditions:} Be a registered user, have previously uploaded training materials, and internet connection.\\*
\textit{Postconditions:} Training materials can be edited.\\*
\textit{Scenario:}
\begin{enumerate}
	\item{Navigate to the content dashboard.}
	\item{To view or edit existing training materials, click on “Manage” within the Training Materials Panel.}
	\item{View the list of all training materials (in plain text), all SMS messages sent, and all quizzes sent.}
	\item{Select “Edit” to edit these existing materials.}
	\item{To add a new training material, click on “Add new” within the Training Materials Panel.}
	\item{Follow use case above for Upload Materials.}
\end{enumerate}
\textit{Exceptions:}
\begin{enumerate}
	\item{Cannot edit texts that have already been sent; must send new version.}\\
\end{enumerate}

\underline{Case 4 - Take Quiz}\\*
\textit{Actor:} Trainee\\*
\textit{Goal:} Respond to quizzes based on specific training materials.\\*
\textit{Preconditions:} Trainee is registered by trainer and has received and completed reading training materials on phone.\\*
\textit{Postconditions:} Trainee can submit responses to quiz.\\*
\textit{Scenario:}
\begin{enumerate}
	\item{Trainee responds to prompt to begin quiz.}
	\item{Receives quiz on phone as SMS message.}
	\item{Responds to questions by texting back his or her answer.}
	\item{Quiz results are sent back to webpage to be parsed and stored in cloud.}
\end{enumerate}
\textit{Exceptions:}
\begin{enumerate}
	\item{Quiz is not available.}
\end{enumerate}


\section{Trainer Interface: The Website}
The following are screenshots of the website we designed and built for the trainers. These four images show the main pages of our website: Login, Home, Manage Users, Training Materials, and Message Log. The goal is a clean, simple interface that is intuitive and functional.

\section{Trainee Interaction: The Phone}
List of texts here

\section{System Design}
The following diagrams show our design visually by specifying the three main components of our system (webpage, cloud storage, and SMS service), the main functionality of each component, and how they interact with each other.

Our system is designed around three main components: a website, SMS, and cloud storage. The website is the way trainers or distributors interact with the system. Through the website, trainers can login to the office of their choosing and manage users, add new training materials and quizzes, and view, edit, and manage previously uploaded training materials and quizzes. To add users, the trainer simply needs to know the trainee's phone number and add it into the system. Adding new training materials and quizzes requires trainers to have these in plain text format and then copy and paste them into the appropriate text fields. From there, the materials can be assigned and sent to trainees via SMS. The website will also allow trainers to view quiz results and accompanying statistical data to measure the learning progress of the employees they train. 

Diagram 1

Trainees will interact with the system through SMS. ...

Diagram 2

Diagram 3

\section{Technologies Used}
\begin{description}
	\item[RapidSMS] a recent UNICEF project built to integrate SMS services with the Django web-framework
	\begin{description}
		\item[Django] a web-framework that lends itself to form collections
		\item[Python] the primary programming language used in Django
	\end{description}
	\item[dotCloud] a cloud web-hosting service that hosts our website in a Python web server and our data in a PostgreSQL database
	\item[Tropo] a web-based SMS service that generates a phone number and API to connect to websites
	\item[GitHub] version control system
\end{description}

\section{Design Rationale}

We chose to use a cloud service for storage rather than a local server. The cloud service holds all of the information that is sent and received between trainers and trainees, including training documents, formatted quizzes, quiz results, and registered users. Storing information “in the cloud” means that it can be accessed by other people directly through the internet, so trainers have access to this information from any computer with internet connection, whether at the office or at home. The cloud offers storage for large amounts of data, so trainers will not be limited by storage space when uploading training materials. The alternative to using cloud storage would be to have all of the training materials stored on a server belonging to the individual social enterprise. We decided against this option because cloud computing is the technology of the future. Although all social enterprises may not have the capability to use cloud services at their offices currently, cloud computing is growing rapidly in popularity and is sure to be utilized by more social enterprises eventually.

Native phone apps were considered, however, there are more barriers to using them. Since our target audience will often have the most basic phones, their phones may not be able to support even java apps that are available on most feature phones, and would require use of the internet which again is often not supported. Even if those problems were not a barrier, the app would need to be installed on the phone and not all phones have a way to install apps via the web and may need to come into the office to install the app. Essentially, there are virtually no startup costs to using SMS and it will reach the broadest audience possible.

The system will also be using a paid SMS service to manage incoming and outgoing texts. While we could write scripts to send texts from an email for free, we must know the user’s service provider and that service provider must have an email domain which may not be extensible to the developing world. There are open source models that work by plugging a gsm modem with a sim card into a computer to receive texts but these are often difficult to work with and hard to scale.

\section{Testing}

\section{User Manual}
